% Options for packages loaded elsewhere
\PassOptionsToPackage{unicode}{hyperref}
\PassOptionsToPackage{hyphens}{url}
%
\documentclass[
]{article}
\usepackage{amsmath,amssymb}
\usepackage{lmodern}
\usepackage{ifxetex,ifluatex}
\ifnum 0\ifxetex 1\fi\ifluatex 1\fi=0 % if pdftex
  \usepackage[T1]{fontenc}
  \usepackage[utf8]{inputenc}
  \usepackage{textcomp} % provide euro and other symbols
\else % if luatex or xetex
  \usepackage{unicode-math}
  \defaultfontfeatures{Scale=MatchLowercase}
  \defaultfontfeatures[\rmfamily]{Ligatures=TeX,Scale=1}
\fi
% Use upquote if available, for straight quotes in verbatim environments
\IfFileExists{upquote.sty}{\usepackage{upquote}}{}
\IfFileExists{microtype.sty}{% use microtype if available
  \usepackage[]{microtype}
  \UseMicrotypeSet[protrusion]{basicmath} % disable protrusion for tt fonts
}{}
\makeatletter
\@ifundefined{KOMAClassName}{% if non-KOMA class
  \IfFileExists{parskip.sty}{%
    \usepackage{parskip}
  }{% else
    \setlength{\parindent}{0pt}
    \setlength{\parskip}{6pt plus 2pt minus 1pt}}
}{% if KOMA class
  \KOMAoptions{parskip=half}}
\makeatother
\usepackage{xcolor}
\IfFileExists{xurl.sty}{\usepackage{xurl}}{} % add URL line breaks if available
\IfFileExists{bookmark.sty}{\usepackage{bookmark}}{\usepackage{hyperref}}
\hypersetup{
  hidelinks,
  pdfcreator={LaTeX via pandoc}}
\urlstyle{same} % disable monospaced font for URLs
\setlength{\emergencystretch}{3em} % prevent overfull lines
\providecommand{\tightlist}{%
  \setlength{\itemsep}{0pt}\setlength{\parskip}{0pt}}
\setcounter{secnumdepth}{-\maxdimen} % remove section numbering
\ifluatex
  \usepackage{selnolig}  % disable illegal ligatures
\fi

\author{}
\date{}

\begin{document}

\hypertarget{header-n0}{%
\section{\texorpdfstring{\(\Sigma\)ignum}{\textbackslash Sigmaignum}}\label{header-n0}}

\hypertarget{header-n2}{%
\subsection{Abstract}\label{header-n2}}

todo

\hypertarget{header-n4}{%
\subsection{Motivation}\label{header-n4}}

:::info\\
Know what you want to do and why that is interesting (maybe with bullet
points). But do not write this section until you know what you actually
have done so that the motivation fits your work.

\begin{verbatim}
○ Establish a research gap:
    ■ What is the problem, the problem space (PD)?
    ■ Why is the problem important that is covered in the thesis? What is the
        problem?
    ■ Why is it hard? What have others done?
○ How do we tackle the problem?
○ What are our hypothesis?
○ What are our techniques? How do you prove that the solution we came up with 
is a GOOD solution? How can you demonstrate that your solution works?
○ What are our findings?
○ (Definition of terms)
○ (Description of remaining chapters)
\end{verbatim}

:::

\begin{itemize}
\item
  research gap

  \begin{itemize}
  \item
    Spezielles Problem: \(sign\big(\sum_i(sign{(x_i)})\big)\)
  \item
    Verbund von zwei Garantien (DP/SMC)

    \begin{itemize}
    \item
      aus vorhandener Froschung zu Datenschutz in maschine learning
      motiviert (DP)
    \item
      Anwendung auf SMC für verteiltes Lernen
    \end{itemize}
  \item
    macht Kooperation von mehreren Instituten mit geschützten Daten
    möglich
  \item
    Schutz gegen

    \begin{itemize}
    \item
      membership inference (und ähnliche) durch DP
    \item
      neidische Institutionen (SMPC)
    \end{itemize}
  \item
    DP wird durch umgebendes Problem gewährleistet
  \item
    in dieser Art ungelöst -\/- das harte ist, das Zwischenergebnis zu
    schützen
  \end{itemize}
\item
  neuer Ansatz mir logischen Verundungen

  \begin{itemize}
  \item
    erfüllt die Anforderungen
  \end{itemize}
\item
  er läuft zuverlässig, ist jedoch sehr teuer
\end{itemize}

\hypertarget{header-n42}{%
\subsection{Background}\label{header-n42}}

:::info\\
You should find and describe related work early on. Know what other
people have done.

\begin{verbatim}
Problem Statement (ca. 1 - 4 pages):
○ Related work (either here or before conclusion):
■ Describe the field in general and how others have tried to solve this
problem
■ In which way is your way better for your hypothesis?
○ Describe in detail the problem you are trying to solve
○ (Hypothesis presentation?)
Optional: Preliminaries
○ Introduce concepts / frameworks that you in your thesis
\end{verbatim}

:::

\begin{itemize}
\item
  SMPC
\item
  Optimizer/maschine learning/...?
\item
  MPyC
\end{itemize}

\hypertarget{header-n55}{%
\subsection{Work Description}\label{header-n55}}

:::info\\
Here you describe the work you have performed, problems you have solved
and methods you have used. There is a fine balance between brevity and
conciseness and ensuring that other people, if investing the time, would
be able to reproduce your results given this description.

\begin{verbatim}
Approach / Methodology:
○ Methodology:
■ How do you solve the problem (described in the problem statement)?
○ System design:
■ Requirements and specifications?
■ Describe how you implemented your approach. If it is a software system
give diagrams, relevant algorithms etc.
○ System implementation
■ Describe the methods that we use, in particular the external methods and
tools (background)■
Describe your approach to solving the problem. Describe any potential
weaknesses of your approach
Experiments:
○ Experimental setup and design choices?:
■ Describe how you implemented the experiments.
■ Talk about the performance of the Azure instance during experiments
○ Experimental implementation:
■ Goal: try to make as concisely clear how you do you what you do
■ Motivate your design choices
■ Describe how you evaluated to show that your approach was successful.
\end{verbatim}

:::

\begin{itemize}
\item
  wir lösen das Problem mit geschickter Umformulierung

  \begin{itemize}
  \item
    Zusammenfassung von \(sign(\sum(b_i)))\) zu
    \(\binom{n}{\lfloor\frac{n}{2}\rfloor+1}\) einzeln überprüfbare
    logische Ausdrücken:

    \begin{itemize}
    \item
      die Ausdrücke bestehen aus jeder Kombination von
      \(\lfloor\frac{n}{2}\rfloor+1\) (etwas mehr als die Hälfte) der
      Elemente, die miteinander verundet sind.
    \item
      Wenn der erste Ausdruck \(1\) ergibt, wird \textbf{1} als Ergebnis
      zurückgegeben und das Evaluieren sofort abgebrochen.
    \item
      Bei negativem Ergebniss aller Läufe wird \textbf{-1}
      zurückgegeben.
    \end{itemize}
  \end{itemize}
\item
  Voraussetzungen sind \(n\) Rechner/Prozesse, die in direkter
  Kommunikation zueinander stehen
\item
  todo: Architektur kurz erläutern
\item
  Performanceanalyse mit Kommandizeilenprogramm \emph{multitime}
\item
  Evaluation durch Test und Gedanken
\end{itemize}

\hypertarget{header-n81}{%
\subsection{\texorpdfstring{Results 🐧 }{Results 🐧 }}\label{header-n81}}

:::info\\
Here you will present and discuss your outcomes: implementation results
or measurements or other project outcomes

\begin{verbatim}
Evaluation / Results:
○ Data analysis
○ Experimental results (objectively describe the results)
Interpretations:
○ Interpretation w.r.t. the hypothesis
\end{verbatim}

:::

\begin{itemize}
\item
  Ergebnisse (Rohdatenbezug -\/- geht nur mit bis zu 11 Teilnehmern
  realistisch, Laufzeit exponentiell,...)
\item
  Interpretation:

  \begin{itemize}
  \item
    geht nur effizient für sehr kleine Personenkreise (\(\le 5\)) -\/-
    dann aber gut
  \end{itemize}
\end{itemize}

\hypertarget{header-n94}{%
\subsection{Conclusion}\label{header-n94}}

:::info\\
Conclusion:\\
○ Summarize your thesis again as in the introduction. Describe how
your\\
evaluation revealed that your system is successful. Describe future work
in \\
this area.\\
Future work:\\
○ Open problems that should be worked on\\
:::

todo

\hypertarget{header-n97}{%
\subsection{Referenzen}\label{header-n97}}

\end{document}
